\documentclass[12pt,a4paper]{article}
\usepackage[spanish,es-tabla]{babel}
\usepackage{bbm}
\usepackage[utf8]{inputenc}
\usepackage{multicol}
\usepackage[T1]{fontenc}
\usepackage{graphicx}
\usepackage{gensymb}
\usepackage{amssymb, amsmath} %Paquetes matemáticos de la American Mathematical Society
\parskip=1 mm
\oddsidemargin -0.8 cm
\headsep -2 cm
\textwidth=17.8cm
\textheight=25.5cm
\begin{document}
\title{Laboratorio de Mecánica, Práctica 9 - Tiro Parabólico}
\date{23 de abril del 2025}
\author{\textbf{Ortega Montero Fernando Naed} - Equipo 4\\
Yibran Morales Munguía\\
Victor Manuel Santillan Romero}
\maketitle
\section{Resumen} 

\section{Introducción}

\section{Desarrollo experimental}

\section{Resultados}

\begin{table}[!h]
\begin{center}
\begin{tabular}{|c|c|}
\hline
tiempo (s) & posición (m) \\ \hline
0.000 (0.001) s & 0.000 (0.001) m \\ \hline
0.033 (0.001) s & 0.085 (0.001) m \\ \hline
0.067 (0.001) s & 0.168 (0.001) m \\ \hline
0.100 (0.001) s & 0.250 (0.001) m \\ \hline
0.133 (0.001) s & 0.330 (0.001) m \\ \hline
0.167 (0.001) s & 0.410 (0.001) m \\ \hline
0.200 (0.001) s & 0.488 (0.001) m \\ \hline
0.233 (0.001) s & 0.567 (0.001) m \\ \hline
0.267 (0.001) s & 0.645 (0.001) m \\ \hline
0.300 (0.001) s & 0.719 (0.001) m \\ \hline
0.333 (0.001) s & 0.802 (0.001) m \\ \hline
0.367 (0.001) s & 0.878 (0.001) m \\ \hline
0.400 (0.001) s & 0.956 (0.001) m \\ \hline
0.433 (0.001) s & 1.030 (0.001) m \\ \hline
0.467 (0.001) s & 1.095 (0.001) m \\ \hline
0.500 (0.001) s & 1.167 (0.001) m \\ \hline
0.533 (0.001) s & 1.248 (0.001) m \\ \hline
0.567 (0.001) s & 1.326 (0.001) m \\ \hline
0.600 (0.001) s & 1.406 (0.001) m \\ \hline
0.633 (0.001) s & 1.488 (0.001) m \\ \hline
0.667 (0.001) s & 1.565 (0.001) m \\ \hline
0.700 (0.001) s & 1.655 (0.001) m \\ \hline
\end{tabular}
\caption{Tiempos - Posiciones del objeto en el eje x}
\end{center}
\end{table}

\begin{table}[!h]
\begin{center}
\begin{tabular}{|c|c|}
\hline
tiempo (s) & posición (m) \\ \hline
0.000 (0.001) s & 0.000 (0.001) m \\ \hline
0.033 (0.001) s & 0.120 (0.001) m \\ \hline
0.067 (0.001) s & 0.226 (0.001) m \\ \hline
0.100 (0.001) s & 0.320 (0.001) m \\ \hline
0.133 (0.001) s & 0.396 (0.001) m \\ \hline
0.167 (0.001) s & 0.464 (0.001) m \\ \hline
0.200 (0.001) s & 0.517 (0.001) m \\ \hline
0.233 (0.001) s & 0.562 (0.001) m \\ \hline
0.267 (0.001) s & 0.597 (0.001) m \\ \hline
0.300 (0.001) s & 0.619 (0.001) m \\ \hline
0.333 (0.001) s & 0.631 (0.001) m \\ \hline
0.367 (0.001) s & 0.633 (0.001) m \\ \hline
0.400 (0.001) s & 0.621 (0.001) m \\ \hline
0.433 (0.001) s & 0.603 (0.001) m \\ \hline
0.467 (0.001) s & 0.574 (0.001) m \\ \hline
0.500 (0.001) s & 0.537 (0.001) m \\ \hline
0.533 (0.001) s & 0.484 (0.001) m \\ \hline
0.567 (0.001) s & 0.421 (0.001) m \\ \hline
0.600 (0.001) s & 0.343 (0.001) m \\ \hline
0.633 (0.001) s & 0.261 (0.001) m \\ \hline
0.667 (0.001) s & 0.161 (0.001) m \\ \hline
0.700 (0.001) s & 0.053 (0.001) m \\ \hline
\end{tabular}
\caption{Tiempos - Posiciones del objeto en el eje y}
\end{center}
\end{table}

\subsection{Calculo de la incertidumbre}

\section{Discusión}

\section{Conclusión}

\section{Referencias}

Miranda, Javier. (2025) \textit{Física Experimental. Introducción a los conceptos y métodos.} Recuperado el 18, 03, 2025, de https://copitarxives.fisica.unam.mx/LT0006ES/LT0006ES.pdf \\

Miranda, Javier. (2000). \textit{EVALUACIÓN DE LA INCERTIDUMBRE EN DATOS EXPERIMENTALES.} \\

Pérez, Héctor. (2018). \textit{Física general.}(Sexta Edición.). México: PATRIA educación \\

\end{document}